%************************************************
\chapter{Introduction}\label{ch:introduction}
%************************************************

It is very likely that you know the following situation: you were at a crowded party and the next day your best friend, who was unable to join, asked you ``How many people where there?''. 
You then struggled to find an answer because you had such intense conversations with the guests that you were unable to put your focus on the other guests.
It turns out that this scene includes many interesting aspects that are relevant for this thesis.
\par
It reminds us that in our daily life we are exposed to situations where multiple events overlap in time.
In the case of the conversation at a party where, is obvious that we can not see all the other guests in crowd of many people.
Concerning our hearing, we know that humans are notably good at performing such a task.
In such a noisy environment we are able to steer our attention to one sound source, even without eye contact and using only a single ear~\cite{bregman90}.
However, this attention mechanisms prevents us from observing the full acoustic scene, even though we would have been able to listen to all sounds.
\par



Research questions
The attenuation of undesired speakers, when multiple concurrent speakers are present, is well known as the ``cocktail party problem''~\cite{haykin05}.
Since almost 70 years (~\cite{cherry53}), research is fascinated by the idea to create a machine extracts the desired sources from a mixture.
This problem is called \emph{source separation} and has many applications for speech and music signals.



Modulations:
The most important observation of speech or music is that sound are not stationary but varies with time and carries the informations.
% DAU
"With most sounds in our environment, such as speech and music, information is contained to a large extent in the changes of sound parameters with time, rather than in the stationary sound segments."

estimating the number of sources might sound simple, but in fact it is very challenging since it is a very abstract problem.

task sounds so simple, yet there so much behind it that we don't. Imagine counting the number trees in a field, compared to counting marbles on a table. Time! Modulation!
% Estimating  the  number  of  people  in  an  image  is  a practical machine vision task that is gaining popularity in the security and surveillance community. 
% on images spatial overlap
% estimating counts i

Scope: We don't want to improve source separation.

Overlap, prevents us from seeing things, thus makes it
Changing the perspective helps.

latent


With the advent of deep learning and AI, research is seeing a paradigm shift 

Compared to vision, audio has a very specific problem 

Now think of a sound object 


We as humans learned to focus on a particular sound source, but in some situation it is very challenging, however, for computers these situation constitute a challenging task.
% Humans are good at classification but not good at extra/interpolation.
% humans are not good a counting larger numbers. So its easy for machines to improve on this. 
Currently we are  a paradigm shift happened. 
Today, machines learning methods can replace hand crafted, engineering researchers often directly 

Do we need to rely on machine learning to solve the task?

paradigma shift

the main questions...


% [ ] Tell a story, and tell it well
% [ ] Tell the reader the problem you are tackling in this project.
% [ ] Quote data sources, e.g., industry analysts, market surveys, case studies 
% [ ] Use plenty of concrete examples (or a running example) and figures 
% [ ] State clearly how you aim to deal with this problem.
% [ ] Limit the scope of your study.

 

\section{Objectives}

The aim of this thesis is \textbf{to investigate the influence of modulations in highly overlapped speech and music signals}.
In order to do this, I outline the following objectives as the basis for this thesis:

\begin{description}
  \item[Scenarios and Datasets:] to develop scenarios where sources are highly overlapped. Also envision scenarios where slow modulations can be exploited. 
  However as real world data often is not available, one objective of this thesis is the creation of new realistic as well as synthetic datasets.
  \item[Representations:] to study and develop representations that allow to improve analysis and processing of these signals.
  \item[Processing Methods:] to develop new methods to address source separation scenarios. These methods would be designed for the constrained scenarios where modulation effects can easily be exploited.
  \item[Generalization:] to transfer the results and insights gained by these controlled studies onto more real-world scenarios.
  \item[Number of Sources:] to investigate and develop new methods to address the task of estimating the number of sources in highly overlapped mixtures.
\end{description}

% \item Can modulations be automatically detected or extracted from highly overlapped signals?
% \item If modulations are known, can they be exploited to improve a task such as source separation?
% \item Can modulations still be utilized when modulation function and its cause of modulation is unknown?
% \item What is the role of modulations in the task of count estimation?



\clearpage
\section{Summary of Contributions}

This thesis comes with five main contributions:

\begin{enumerate}
\item I discuss scenarios of time and frequency overlapped audio sources.
I consider known scenarios for speech and music but also present a novel scenario where instruments are playing in unison.
In this scenario, I show, how slowly varying tempo-spectral modulations, caused e.g. by vibrato, can be utilized for separation and source count estimation of highly overlapped signals.
Furthermore I show how these scenarios can stimulate new research directions to \emph{analyze} and \emph{process} such signals.\\

\item I developed two novel methods to \emph{separate} unison instrument mixtures: one is informed by an estimate of the fundamental frequency variation.
The other is unsupervised, inspired by the way how humans segregate time-varying sources.
Finally, I study how the observations from the unison scenario can be transferred to real world scenarios such as the separation of professional produced music.\\

\item I provide two detailed experimental studies to assess how humans perceive highly overlapped mixtures.
In these studies I focussed on the \emph{number of concurrent sources} in scenarios such as a overlapped speech as well as polyphonic music recordings.
In this vein, I present the results of auditory experiments that studied the humans ability to detect the maximum number of sources.\\

\item For the task of \emph{estimating the maximum number of concurrent} speakers I developed a method based on deep neural networks that addresses cocktail party like environments.
Furthermore, I show that this model reached state-of-the art performance when compared to other models and also supersedes human performance when compared with the results of subjective listening experiments.
Finally I revealed the relation between slow modulations in speech and the ability of a model to count.\\

\item Last, but not least, I provided service to the research community by co-organizing the international source separation evaluation campaign (SiSEC) and helped out to improve sustainability and reproducibility in our field.
In that vein, I provided open software to assess the quality of separation system. 
These tools were also used throughout the work presented in this thesis.
\end{enumerate}

Since this work is based on these publications, they are cited repeatedly throughout this thesis. However, for readability reasons, if a section is mainly based on one of these publication, a remark is added at side of the section, instead of citing the same publication exhaustively. Furthermore, it should be note that these contributions are based on the ideas and research of the author of this thesis, and is reflected in the publications by being the lead author.

\section{Structure of this Thesis}

The thesis and its relevant linked publications are organized into 6 main chapters.
\begin{description}
  \item[Chapter 2] explains the fundamental concepts of audio signals (Section~\ref{sec:specifics-of-audio-signals}) as well as sources and overlapped sounds (Section~\ref{sources-and-mixtures}), relevant for the remainder of this thesis. 
  This includes commonly used transformations and signal representations for the task of source separation and source count estimation.
  Furthermore, the process of mixing sound sources as well as its inverse task --- sound source separation --- are explained.
  The chapter also covers basics of fundamental frequency and its variations (Section~\ref{sub:time-variant-audio-signals}) as an important feature for harmonic audio signals.
  Part of this chapter is based on~\cite{rafii18}.
  \item[Chapter 3] introduces relevant tasks and applications in the context of highly overlapped sounds.
  Furthermore, the relevance of highly overlapped source scenarios are discussed and a new research track of slow  modulations (Section~\ref{exploiting-slow-modulations}) is proposed.
  \item[Chapter 4] presents and discusses the importance of data for analysis and evaluation.
  In this chapter synthetic (Section~\ref{}) and realistic (Section~\ref{}) datasets are presented, created during the course of this thesis~\cite{oss_wice, oss_unison, oss_libricount, liutkus17}.
  \item[Chapter 5 and 6] presents separation methods that are developed over the course of this thesis. 
  The covers techniques that utilize modulation information when available(Chapter 5, ~\cite{stoeter14, stoeter15acm, stoeter15icassp}) as well as blind methods (Chapter 6~\cite{stoeter16, liutkus17}).
  \item[Chapter 7 and 8] covers the analysis of overlapped sounds. Specifically, it deals with identifying and estimating the number of sources on music~\cite{schoeffler13, stoeter13} and the cocktail party scenario~\cite{stoeter19, stoeter18}.
  \item[Chapter 9] Concludes this thesis and gives a and outlook into future research directions.
\end{description}
