%*******************************************************
% Abstract
%*******************************************************
%\renewcommand{\abstractname}{Abstract}
\pdfbookmark[1]{Abstract}{Abstract}
% \addcontentsline{toc}{chapter}{\tocEntry{Abstract}}
\begingroup
\let\clearpage\relax
\let\cleardoublepage\relax
\let\cleardoublepage\relax

\chapter*{Abstract}
Everyday audio recordings involve mixture signals: music contains a mixture of instruments; in a meeting or conference, there is a mixture of human voices. 
For these mixtures, automatically separating or estimating the number of sources is a challenging task.
A common assumption when processing mixtures in the time-frequency domain is that sources are not fully overlapped.
However, in this work we consider some cases where the overlap is severe --- for instance, when instruments play the same note (unison) or when many people speak concurrently ("cocktail party") --- highlighting the need for new representations and more powerful models.
\par
To address the problems of source separation and count estimation, we use conventional signal processing techniques as well as deep neural networks (DNN).
We first address the source separation problem for unison instrument mixtures, studying the distinct spectro-temporal modulations caused by vibrato.
To exploit these modulations, we developed a method based on time warping, informed by an estimate of the fundamental frequency.
For cases where such estimates are not available, we present an unsupervised model, inspired by the way humans group time-varying sources (common fate).
This contribution comes with a novel representation that improves separation for overlapped and modulated sources on unison mixtures but also improves vocal and accompaniment separation when used as an input for a DNN model.
\par
Then, we focus on estimating the number of sources in a mixture, which is important for real-world scenarios.
Our work on count estimation was motivated by a study on how humans can address this task, which lead us to conduct listening experiments, confirming that humans are only able to estimate the number of up to four sources correctly.
To answer the question of whether machines can perform similarly, we present a DNN architecture, trained to estimate the number of concurrent speakers.
Our results show improvements compared to other methods, and the model even outperformed humans on the same task.
\par
In both the source separation and source count estimation tasks, the key contribution of this thesis is the concept of ``modulation'', which is important to computationally mimic human performance. 
Our proposed Common Fate Transform is an adequate representation to disentangle overlapping signals for separation, and an inspection of our DNN count estimation model revealed that it proceeds to find modulation-like intermediate features.
\vfill

\begin{otherlanguage}{ngerman}
\pdfbookmark[1]{Zusammenfassung}{Zusammenfassung}
\chapter*{Zusammenfassung}

Im Alltag sind wir von gemischten Signalen umgeben: Musik besteht aus einer Mischung von Instrumenten; in einem Meeting oder auf einer Konferenz sind wir einer Mischung menschlicher Stimmen ausgesetzt.
Für diese Mischungen ist die automatische Quellentrennung oder die Bestimmung der Anzahl an Quellen eine anspruchsvolle Aufgabe.
Eine häufige Annahme bei der Verarbeitung von gemischten Signalen im Zeit-Frequenzbereich ist, dass die Quellen sich nicht vollständig überlappen.
In dieser Arbeit betrachten wir jedoch einige Fälle, in denen die Überlappung immens ist --- zum Beispiel, wenn Instrumente den gleichen Ton spielen (unisono) oder wenn viele Menschen gleichzeitig sprechen (Cocktailparty) ---, so dass neue Signal-Repräsentationen und leistungsfähigere Modelle notwendig sind.
\par
Um die zwei genannten Probleme zu bewältigen, verwenden wir sowohl konventionelle Signalverbeitungsmethoden als auch tiefgehende neuronale Netze (DNN).
Wir gehen zunächst auf das Problem der Quellentrennung für Unisono-Instrumentenmischungen ein und untersuchen die speziellen, durch Vibrato ausgelösten, zeitlich-spektralen Modulationen.
Um diese Modulationen auszunutzen entwickelten wir eine Methode, die auf Zeitverzerrung basiert und eine Schätzung der Grundfrequenz als zusätzliche Information nutzt.
Für Fälle, in denen diese Schätzungen nicht verfügbar sind, stellen wir ein unüberwachtes Modell vor, das inspiriert ist von der Art und Weise, wie Menschen zeitveränderliche Quellen gruppieren (Common Fate).
Dieser Beitrag enthält eine neuartige Repräsentation, die die Separierbarkeit für überlappte und modulierte Quellen in Unisono-Mischungen erhöht, aber auch die Trennung in Gesang und Begleitung verbessert, wenn sie  in einem DNN-Modell verwendet wird.
\par
Im Weiteren beschäftigen wir uns mit der Schätzung der Anzahl von Quellen in einer Mischung, was für reale Szenarien wichtig ist.
Unsere Arbeit an der Schätzung der Anzahl war motiviert durch eine Studie, die zeigt, wie wir Menschen diese Aufgabe angehen.
Dies hat uns dazu veranlasst, eigene Hörexperimente durchzuführen, die bestätigten, dass Menschen nur in der Lage sind, die Anzahl von bis zu vier Quellen korrekt abzuschätzen.
Um nun die Frage zu beantworten, ob Maschinen dies ähnlich gut können, stellen wir eine DNN-Architektur vor, die erlernt hat, die Anzahl der gleichzeitig sprechenden Sprecher zu ermitteln.
Die Ergebnisse zeigen Verbesserungen im Vergleich zu anderen Methoden, aber vor allem auch im Vergleich zu menschlichen Hörern.
\par
Sowohl bei der Quellentrennung als auch bei der Schätzung der Anzahl an Quellen ist ein Kernbeitrag dieser Arbeit das Konzept der ``Modulation'', welches wichtig ist, um die Strategien von Menschen mittels Computern nachzuahmen.
Unsere vorgeschlagene Common Fate Transformation ist eine adäquate Darstellung, um die Überlappung von Signalen für die Trennung zugänglich zu machen und eine Inspektion unseres DNN-Zählmodells ergab schließlich, dass sich auch hier modulationsähnliche Merkmale finden lassen.
\end{otherlanguage}

\endgroup

\vfill
