%\vspace{-0.2cm}
\chapter{Datasets for Separation and Analysis}

Building a good data-driven method for source separation relies heavily on a training dataset to learn the separation model. In our case, this not only means obtaining a set of musical songs, but also their constitutive accompaniment and lead sources, summing up to the mixtures. For professionally-produced or recorded music, the separated sources are often either unavailable or private. Indeed, they are considered amongst the most precious assets of right holders, and it is very difficult to find isolated vocals and accompaniment of professional bands that are freely available for the research community to work on without copyright infringements.

Another difficulty arises when considering that the different sources in a musical content do share some common orchestration and are not superimposed in a random way, prohibiting simply summing isolated random notes from instrumental databases to produce mixtures. This contrasts with the speech community which routinely generates mixtures by summing noise data~\cite{varga93} and clean speech~\cite{garofolo93}.

 Furthermore, the temporal structures in music signals typically spread over long periods of time and can be exploited to achieve better separation. Additionally, short excerpts do not often comprise parts where the lead signal is absent, although a method should learn to deal with that situation. This all suggests that including full songs in the training data is preferable over short excerpts.

Finally, professional recordings typically undergo sophisticated sound processing where panning, reverberation, and other sound effects are applied to each source separately, and also to the mixture. To date, simulated data sets have poorly mimicked these effects~\cite{sturmel12}. Many separation methods make  assumptions about the mixing model of the sources, e.g., assuming it is linear (i.e., does not comprise effects such as dynamic range compression). It is quite common that methods giving extremely good performance for linear mixtures completely break down when processing published musical recordings. Training and test data should thus feature realistic audio engineering to be useful for actual applications.

In this context, the development of datasets for lead and accompaniment separation was a long process. In early times, it was common for researchers to test their methods on some private data. To the best of our knowledge, the first attempt at releasing a public dataset for evaluating vocals and accompaniment separation was the Music Audio Signal Separation (MASS) dataset \cite{MTGMASSdb}. %, released in 2008.
It strongly boosted research in the area, even if it only featured 2.5 minutes of data. The breakthrough was made possible by some artists which made their mixed-down audio, as well as its constitutive stems (unmixed tracks), available under open licenses such as Creative Commons, or authorized scientists to use their material for research.

The MASS dataset then formed the core content of the early Signal Separation Evaluation Campaigns (SiSEC) \cite{vincent09}, which evaluate the quality of various music separation methods \cite{araki10,araki12,vincent12,ono15,liutkus17}. SiSEC always had a strong focus on vocals and accompaniment separation. For a long time, vocals separation methods were very demanding computationally and it was already considered extremely challenging to separate excerpts of only a few seconds.

In the following years, new datasets were proposed that improved over the MASS dataset in many directions. We briefly describe the most important ones, summarized in Table~\ref{tab:datasets}.
\begin{itemize}[leftmargin=*]
	\item The QUASI dataset was proposed to study the impact of different mixing scenarios on the separation quality. It  consists of the same tracks as in the MASS dataset, but kept full length and mixed by professional sound engineers.
	\item The MIR-1K and iKala datasets were the first attempts to scale vocals separation up. They feature a higher number of samples than the previously available datasets. However, they consist of mono signals of very short and amateur karaoke recordings.
	\item The ccMixter dataset was proposed as the first dataset to feature many full-length stereo tracks. Each one comes with a vocals and an accompaniment source. Although it is stereo, it often suffers from simplistic mixing of sources, making it unrealistic in some aspects.
	\item MedleyDB has been developed as a dataset to serve many purposes in music information retrieval. It consists of more than 100 full-length recordings, with all their constitutive sources. It is the first dataset to provide such a large amount of data to be used for audio separation research (more than 7 hours). Among all the material present in that dataset, 63 tracks feature singing voice.
  \item DSD100 was presented for SiSEC 2016. It features 100 full-length tracks originating from the 'Mixing Secret' Free Multitrack Download Library\footnote{\url{http://www.cambridge-mt.com/ms-mtk.htm}} of the Cambridge Music Technology, which is freely usable for research and educational purposes.
\end{itemize}

Finally, we present here the MUSDB18 dataset, putting together tracks from MedleyDB, DSD100, and other new musical material. It features 150 full-length tracks, and has been constructed by the authors of this paper so as to address all the limitations we identified above:
\begin{itemize}[leftmargin=*]
\item It only features full-length tracks, so that the handling of long-term musical structures, and of silent regions in the lead/vocal signal, can be evaluated.
\item It only features stereo signals which were mixed using professional digital audio workstations. This results in quality stereo mixes which are representative of real application scenarios.
\item As with DSD100, a design choice of MUSDB18 was to split the signals into 4 predefined categories: bass, drums, vocals, and other. This contrasts with the enhanced granularity of MedleyDB that offers more types of sources, but it strongly promotes automation of the algorithms.
\item Many musical genres are represented in MUSDB18, for example, jazz, electro, metal, etc.
\item It is split into a development (100 tracks, 6.5~h) and a test dataset (50 tracks, 3.5~h), for the design of data-driven separation methods.
\end{itemize}

All details about this freely available dataset and its accompanying software tools may be found in its dedicated website\footnote{https://sigsep.github.io/musdb}.

In any case, it can be seen that datasets of sufficient duration to build data-driven separation methods were only created recently.

\begin{table*}[htbp]
	\centering
	\caption{Summary of datasets available for lead and accompaniment separation. Tracks without vocals were omitted in the statistics.}
	\label{tab:datasets}
		\begin{tabular}{|l l l l l l l|}
			\hline
			\textbf{Dataset} & \textbf{Year} & \textbf{Reference(s)} & \textbf{URL} & \textbf{Tracks} & \textbf{Track duration (s)} & \textbf{Full/stereo?}\\
			\hline
			MASS & 2008 & \cite{MTGMASSdb} & \url{http://www.mtg.upf.edu/download/datasets/mass} & 9 & $16 \pm 7$ & no / yes \\
			MIR-1K & 2010 & \cite{hsu10} & \url{https://sites.google.com/site/unvoicedsoundseparation/mir-1k} & 1,000 & $8 \pm 8$ & no / no \\
			QUASI & 2011 & \cite{liutkus11,vincent12} & \url{http://www.tsi.telecom-paristech.fr/aao/en/2012/03/12/quasi/} & 5 & $206 \pm 21$ & yes / yes \\
			ccMixter & 2014 & \cite{liutkus142} & \url{http://www.loria.fr/~aliutkus/kam/} & 50 & $231 \pm 77 $ & yes / yes \\
			MedleyDB & 2014 & \cite{bittner14} & \url{http://medleydb.weebly.com/} & 63 & $206 \pm 121$ & yes / yes \\
			iKala & 2015 & \cite{chan15} & \url{http://mac.citi.sinica.edu.tw/ikala/} & 206 & 30 & no / no \\
			DSD100 & 2015 & \cite{ono15} & \url{sisec17.audiolabs-erlangen.de} & 100 & $251 \pm 60$ & yes / yes \\
      MUSDB18 & 2017 & \cite{rafii17} & \url{https://sigsep.github.io/musdb} & 150 & $236 \pm 95$ & yes / yes \\
			\hline
		\end{tabular}
\end{table*}


%\vspace{-0.1cm}
\section{SiSEC Evaluation Campaigns}
\label{sec:evaluation}

\subsection{Organisation and Open Source Tools}
\label{ssec:background}

The problem of evaluating the quality of audio signals is a research topic of its own, which is deeply connected to psychoacoustics~\cite{zwicker13} and has many applications in engineering because it provides an objective function to optimize when designing processing methods. While mean squared error (MSE) is often used for mathematical convenience whenever an error is to be computed, it is a very established fact that MSE is not representative of audio perception~\cite{rix01,wang09}. For example, inaudible phase shifts would dramatically increase the MSE. Moreover, it should be acknowledged  that the concept of quality is rather application-dependent.

In the case of signal separation or enhancement, processing is often only a part of a whole architecture and a relevant methodology for evaluation is to study the positive or negative impact of this module on the overall performance of the system, rather than to consider it independently from the rest. For example, when embedded in an automatic speech recognition (ASR) system, performance of speech denoising can be assessed by checking whether it decreases word error rate~\cite{barker15}.

When it comes to music processing, and more particularly to lead and accompaniment separation, the evaluation of separation quality has traditionally been inspired by work in the audio coding community~\cite{recommendation2001MUSHRA,rix01} in the sense that it aims at comparing ground truth vocals and accompaniment with their estimates, just like audio coding compares the original with the compressed signal.

%\vspace{-0.2cm}
\subsection{Metrics}
\label{ssec:metrics}

As noted previously, MSE-based error measures are not perceptually relevant. For this reason, a natural approach is to have humans do the comparison. The gold-standard for human perceptual studies is the MUlti Stimulus test with Hidden Reference and Anchor (MUSHRA) methodology, that is commonly used for evaluating audio coding~\cite{recommendation2001MUSHRA}.

However, it quickly became clear that the specific evaluation of separation quality cannot easily be reduced to a single number, even when achieved through actual perceptual campaigns, but that quality rather depends on the application considered. For instance, karaoke or vocal extraction come with opposing trade-offs between isolation and distortion. For this reason, it has been standard practice to provide different and complementary metrics for evaluating separation that measure the amount of distortion, artifacts, and interference in the results.

While human-based perceptual evaluation is definitely the best way to assess separation quality~\cite{vincent062,cano11}, having computable objective metrics is desirable for several reasons. First, it allows researchers to evaluate performance without setting up costly and lengthy perceptual evaluation campaigns. Second, it permits large-scale training for the fine-tuning of parameters. In this respect, the Blind Source Separation Evaluation (BSS Eval) toolbox~\cite{fevotte05a,vincent06} provides quality metrics in decibel to account for distortion (SDR), artifacts (SAR), and interferences (SIR). Since it was made available quite early and provides somewhat reasonable correlation with human perception in certain cases \cite{fox07,fox072} it is still widely used to this day.

Even if BSS Eval was considered sufficient for evaluation purposes for a long time, it is based on squared error criteria. Following early work in the area~\cite{kornycky08}, the Perceptual Evaluation of Audio Source Separation (PEASS) toolkit \cite{emiya10,emiya11,vincent122} was introduced as a way to predict perceptual ratings. While the methodology is very relevant, PEASS however was not widely accepted in practice. We believe this is for two reasons. First, the proposed implementation is quite computationally demanding. Second, the perceptual scores it was designed with are more related to speech separation than to music.
%Derry: Do we have space to add references to work which shows that PEASS fails to correlate with perception for many separation tasks?

Improving perceptual evaluation often requires a large amount of experiments, which is both costly and requires many expert listeners. One way to increase the number of participants is to conduct web-based experiments. In~\cite{cartwright16}, the authors report they were able to aggregate 530 participants in only 8.2 hours and obtained perceptual evaluation scores comparable to those estimated in the controlled lab environment.

Finally, we highlight here that the development of new perceptually relevant objective metrics for singing voice separation evaluation remains an open issue~\cite{gupta15}. It is also a highly crucial one for future research in the domain.
